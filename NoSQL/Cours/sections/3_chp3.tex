\section{Concepts fondamentaux}

% \begin{frame}{Caractéristiques principales du NoSQL}

%     \begin{itemize}
%         \item Schéma flexible
%         \item Scalabilité
%         \item Haute disponibilité
%         \item Réplication
%         \item Partitionnement
%         \item Indexation
%         \item Transactions
%     \end{itemize}
% \end{frame}

% \begin{frame}{Schéma flexible}

%     \begin{itemize}
%         \item Schéma fixe : SQL
%         \item Schéma flexible : NoSQL
%     \end{itemize}

% \end{frame}

% \begin{frame}{Scalabilité}

%     Scalabilité horizontale : Ajout de machines pour augmenter la capacité de stockage et de traitement

%     Scalabilité verticale : Ajout de ressources (CPU, RAM, Disque) sur une machine existante

% \end{frame}

% \begin{frame}{Haute disponibilité}

%     Garantir un service disponible en tout temps

%     \begin{itemize}
%         \item Réplication
%         \item Partitionnement
%         \item Tolérance aux pannes
%     \end{itemize}

% \end{frame}

% \begin{frame}{Réplication}

%     \begin{itemize}
%         \item Réplication synchrone
%         \item Réplication asynchrone
%         \item Réplication multi-maître
%     \end{itemize}

% \end{frame}

% \begin{frame}{Partitionnement}

%     Partitionnement ou sharding : séparer les données en plusieurs chunks

%     Faire diagramme avec les différentes stratégies de partitionnement

%     \begin{itemize}
%         \item Partitionnement par clé
%         \item Partitionnement par plage
%         \item Partitionnement par hachage
%     \end{itemize}

% \end{frame}

\begin{frame}{Sharding (partitionnement horizontal)}

    % \begin{tikzpicture}[line width=1pt]
    %     \node[database,label=below:DB,database radius=1cm,database segment height=0.5cm] at (3,0) {};
    % \end{tikzpicture}

    \begin{tikzpicture}[line width=1pt]
        \node[database,ultra thick,database radius=1cm,database segment height=0.5cm, database top segment={draw=black,fill=customgreen}, database middle segment={draw=black,fill=customgreen}, database bottom segment={draw=black,fill=customgreen}] at (0,0) {};
        \node[database,ultra thick,database radius=1cm,database segment height=0.5cm, database top segment={draw=black,fill=customblue}, database middle segment={draw=black,fill=customblue}, database bottom segment={draw=black,fill=customblue}] at (3,0) {};
        \node[database,ultra thick,database radius=1cm,database segment height=0.5cm, database top segment={draw=black,fill=customred}, database middle segment={draw=black,fill=customred}, database bottom segment={draw=black,fill=customred}] at (6,0) {};

    \end{tikzpicture}

\end{frame}

\begin{frame}{Sharding (partitionnement horizontal)}

    \begin{tikzpicture}[line width=1pt]
        \node[database,ultra thick,database radius=1cm,database segment height=0.5cm, database top segment={draw=black,fill=customgreen}, database middle segment={draw=black,fill=customgreen}, database bottom segment={draw=black,fill=customgreen}] at (0,0) {};
        \node[database,ultra thick,database radius=1cm,database segment height=0.5cm, database top segment={draw=black,fill=customblue}, database middle segment={draw=black,fill=customblue}, database bottom segment={draw=black,fill=customblue}] at (3,0) {};
        \node[database,ultra thick,database radius=1cm,database segment height=0.5cm, database top segment={draw=black,fill=customred}, database middle segment={draw=black,fill=customred}, database bottom segment={draw=black,fill=customred}] at (6,0) {};

        \node[database,ultra thick,database radius=1cm,database segment height=0.5cm, database top segment={draw=black,fill=customgreen}, database middle segment={draw=black,fill=customblue}, database bottom segment={draw=black,fill=customred}] at (3,-3) {};
    \end{tikzpicture}

    
\end{frame}

\begin{frame}{Round Robin}

    \begin{tikzpicture}
        
        % Include the file stack icon
        \node[anchor=south west,scale=0.6] at (0,0) {\filestack};
        
        \node[
            database,ultra thick,database radius=1cm,database segment height=0.5cm, 
            database top segment={draw=black,fill=customgreen}, 
            database middle segment={draw=black,fill=customgreen}, 
            database bottom segment={draw=black,fill=customgreen}
        ] at (2,2) {};

        \node[
            disk,ultra thick,disk radius=1cm,disk segment height=0.5cm,
        ] at (5,-2) {};
    
    \end{tikzpicture}
    
\end{frame}

% \begin{frame}{Indexation}

%     \begin{itemize}
%         \item Indexation secondaire
%         \item Indexation composite
%         \item Indexation géospatiale
%     \end{itemize}

% \end{frame}


% \begin{frame}{Avantages et inconvénients du NoSQL}

% Avantages : Scalabilité, Flexibilité, Performances pour certaines charges de travail

% Inconvénients : Complexité de gestion, Manque de standardisation, Consistance éventuelle (CAP Theorem)

% \end{frame}


% \begin{frame}{Théorème CAP}

%     \begin{figure}[htb]
%         \resizebox{0.85\textwidth}{!}{
%         \begin{tikzpicture}        
%             % Circle 1: Consistency
%             \fill[pattern=north east lines, pattern color=customblue] (0,0) circle (3cm);
%             \draw[very thick, color=customblue] (0,0) circle (3cm);
%             \node[fill=customblue, text=white, font=\bfseries, rounded corners] at (0,0) {Consistency};
        
%             % Circle 2: Availability
%             \fill[pattern=north east lines, pattern color=customred] (-2.5,-3.25) circle (3cm);
%             \draw[very thick, color=customred] (-2.5,-3.25) circle (3cm);
%             \node[fill=customred, text=white, font=\bfseries, rounded corners] at (-2.5,-3.25) {Availability};
        
%             % Circle 3: Partition Tolerance
%             \fill[pattern=north east lines, pattern color=customgreen] (2.5,-3.25) circle (3cm);
%             \draw[very thick, color=customgreen] (2.5,-3.25) circle (3cm);
%             \node[fill=customgreen, text=white, font=\bfseries, rounded corners] at (2.5,-3.25) {Partition tolerance};
            
%             \onslide<2->{
%             \node[inner sep=0pt] (unicorn) at (0,-2.5) {\includegraphics[width=1cm]{img/unicorn.png}};
%             }
%         \end{tikzpicture}
%         }
%     \end{figure}
% \end{frame}


% \begin{frame}{Théorème CAP}

%     \begin{itemize}
%         \item Consistency : Toutes les données sont à jour
%         \item Availability : Toutes les requêtes reçoivent une réponse
%         \item Partition tolerance : Le système continue de fonctionner malgré les partitions réseau
%     \end{itemize}

% \end{frame}

% \begin{frame}{Cas d'usage des bases de données NoSQL}

%     \begin{itemize}
%         \item Applications web et mobiles
%         \item Big Data et analytics
%         \item Gestion de contenu et réseaux sociaux
%         \item Internet des objets (IoT)
%     \end{itemize}

% \end{frame}

% \begin{frame}[standout]
    
% Time for Kahoot!

% \end{frame}