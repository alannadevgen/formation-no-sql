\section{Introduction au NoSQL}

\begin{frame}{Pourquoi NoSQL ?}

    \begin{itemize}
        \item Besoin de stocker et de traiter des données massives
        \item Besoin de scalabilité
        \item Besoin de haute disponibilité
        \item Besoin de réplication
        \item Besoin de partitionnement
        \item Besoin d'indexation
        \item Besoin de transactions
    \end{itemize}

\end{frame}

\begin{frame}{Mais que signifie NoSQL ?}

    NoSQL signifie Not Only SQL ou No SQL.

\end{frame}

\begin{frame}{Définition}

    Johan Oskarsson définit NoSQL comme un terme générique pour désigner les bases de données qui ne sont pas des bases de données relationnelles.

\end{frame}


\begin{frame}{NewSQL}

    NewSQL est un terme générique pour désigner les bases de données qui sont des bases de données relationnelles mais qui sont conçues pour être distribuées.

    Examples : Google Spanner, CockroachDB, NuoDB, VoltDB, MemSQL, Clustrix, etc.
\end{frame}

\begin{frame}[standout]
    
Time for Kahoot!

\end{frame}